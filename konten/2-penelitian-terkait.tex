\section{Penelitian Terkait}
\label{sec:penelitianterkait}

Beberapa penelitian sebelumnya telah berhasil dalam mengembangkan lingkungan simulasi untuk robot menggunakan ROS (Pendahulu ROS 2) dan Gazebo.
Seperti yang dilakukan Qian et al. \citep{qian2014} yang mengembangkan simulasi untuk robot \emph{manipulator}, Zhang et al. \citep{zhang2015} yang mengembangkan simulasi untuk robot \emph{quadrotor UAV}, dan Takaya et al. \citep{takaya2016} yang mengembangkan lingkungan simulasi untuk pengujian terhadap \emph{mobile robot}.
Namun, berbeda dengan penelitian yang telah dilakukan sebelumnya, penelitian yang akan kami lakukan memilih menggunakan ROS 2 agar kontroler robot yang dibuat untuk simulasi memiliki performa yang lebih baik serta dapat bekerja secara \emph{real-time} \citep{maruyama2016}.

Selain itu, penelitian lain juga telah dilakukan oleh Erickson et al. \citep{erickson2020} yang mengembangkan Assistive Gym, sebuah \emph{framework} simulasi untuk \emph{assistive robotics} berbasis OpenAI Gym.
\emph{Framework} simulasi tersebut kemudian digunakan oleh Clegg et al. \citep{clegg2020} untuk mengembangkan metode \emph{learning} melalui simulasi pada kolaborasi antara robot dengan manusia dalam membantu pemakaian baju pada manusia.
Namun, karena tidak menggunakan ROS, kontroler robot yang dibuat untuk simulasi yang menggunakan \emph{framework} tersebut perlu dibuat ulang ketika akan diujikan secara langsung pada pengguna menggunakan robot fisik.
Walaupun begitu, penelitian yang dilakukan oleh Zamora et al. \citep{zamora2016} menunjukkan bahwa simulasi yang ada pada OpenAI Gym juga bisa diintegrasikan pada ROS dan Gazebo, sehingga tidak menutup kemungkinan bahwa Assistive Gym juga bisa digunakan bersamaan dengan ROS 2 dan Gazebo.
